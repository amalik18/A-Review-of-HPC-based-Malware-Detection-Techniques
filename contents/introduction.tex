\documentclass[../main.tex]{subfiles}
\usepackage[utf8x]{inputenc}
\usepackage{blindtext}
\usepackage{float}
\usepackage{graphicx}
\usepackage{siunitx}
\usepackage{cite}

\begin{document}

As data security has progressed, as to have malicious actors. With the great digital revolution, access to the internet has become ubiquitous which has led to billions of users. These billions of users present a very lucrative target for a malicious actor. All it takes is an uneducated user falling victim to something as simple as Adware to compromise their system and leave their precious data for the taking. 

Luckily, security measures have progresses significantly and they continue to progress even further. Gone are the days of solely relying on signature-based detection. The proliferation of computers, in conjunction with improvements to CPUs and GPUs has given arise to newer detection methods. Since there are numerous detection methods, this paper will focus solely on HPC-based malware detection.

\end{document}