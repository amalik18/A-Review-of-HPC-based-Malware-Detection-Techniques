\documentclass[../main.tex]{subfiles}
\usepackage[utf8x]{inputenc}
\usepackage{blindtext}
\usepackage{float}
\usepackage{graphicx}
\usepackage{siunitx}
\usepackage{cite}

\begin{document}

Before moving to reviewing the various studies, lets first showcase some challenges in research and implementation of this topic. 

Regarding research into HPCs, it's increasingly difficult to predict the outcome of HPC. The non-deterministic nature of HPCs in general make it very hard to come to a completely deterministic conclusion. In addition, because of the limited number of HPCs on a chip, ranging from 4 and 20, it becomes a challenge to efficiently collect many event statistics \cite{memory}. The only events that can come to a deterministic result are the varying number of \textit{retired instruction counters}, because they only depend on the underlying instruction set architecture (ISA) and, in theory, should be stable between runs \cite{determinism}. Even then, these events can still be non-deterministic due to things like \cite{determinism}:
\begin{itemize}
    \item OS interactions
    \item Hardware implementation
    \item Processor variation
\end{itemize}

All of this is possible even in an environment designed, specifically, to give deterministic results, SPEC CPU. \\
Due to this very fact, it becomes increasingly difficult to design and run repeatable tests to create meaningful data. In a study conducted to validate retired instruction performance counter data from the SPEC CPU 2000 and 2006 benchmark suites on various different implementations of the x86 architecture, it was found that even intra-machine results varied between x86 implementations \cite{4636099}.

\end{document}