\documentclass[../main.tex]{subfiles}
\usepackage[utf8x]{inputenc}
\usepackage{blindtext}
\usepackage{float}
\usepackage{graphicx}
\usepackage{siunitx}
\usepackage{cite}

\begin{document}

As shown, HPCs have become increasingly important in the world of malware detection and prevention. As malware becomes more and more complex, Anti-Virus Software is unable to keep up and provide strong defense against them \cite{time_series}. This is where we can improve detection by employing HPCs, however, the limitation of "one-event-one-counter" is the bottleneck. The introduction of multiplexing within HPCs can improve how many events a single HPC can count \cite{memory}. In addition, by using approximate counting algorithms we can also reduce the storage footprint of the HPCs \cite{memory}. 

As technology progresses, code gets more complex, systems become more intelligent, it becomes necessary to invent novel approaches to malware detection. One instance is the usage of HPCs, however, more is required. Great work has been done in the realm of Machine Learning based HPC malware detection, it should hope to be fruitful. This is not to say that HPCs are perfect, as noted earlier, the non-deterministic nature of HPCs makes it incredibly difficult to reproduce events and counters \cite{sok}. 

The increased usage of HPCs should lend itself to various areas from enterprise to home, from rack-mounted servers to embedded IoT devices. Various types of attack vectors can be addressed using HPCs, including Control Hijacking Attacks \cite{control}.

\end{document}