\documentclass[conference]{IEEEtran}
\IEEEoverridecommandlockouts
% The preceding line is only needed to identify funding in the first footnote. If that is unneeded, please comment it out.
\usepackage{cite}
\usepackage{amsmath,amssymb,amsfonts}
\usepackage{algorithmic}
\usepackage{graphicx}
\usepackage{textcomp}
\usepackage{xcolor}
\usepackage{blindtext}
\usepackage{subfiles} 

\def\BibTeX{{\rm B\kern-.05em{\sc i\kern-.025em b}\kern-.08em
    T\kern-.1667em\lower.7ex\hbox{E}\kern-.125emX}}
\begin{document}

\title{A Review of HPC-based Malware Detection Techniques\\}

\author{\IEEEauthorblockN{Ali Malik}
\IEEEauthorblockA{\textit{Volgenau School of Engineering} \\
\textit{George Mason University}\\
Fairfax, Virginia, USA \\
amalik18@gmu.edu}
}

\maketitle

\begin{abstract}
\subfile{contents/abstract.tex}
\end{abstract}

\begin{IEEEkeywords}
hardware performance counters, malware detection
\end{IEEEkeywords}

\section{Introduction}
\subfile{contents/introduction.tex}

\section{Research Challenges}
\subfile{contents/research_challenges.tex}

\section{The Survey}
In the next few sections we'll take a look at prominent research papers and tie them together to form a larger story.

\subsection{Theoretical Study of HPCs}\label{T}
\subfile{contents/theoretical.tex}

\subsection{Memory Efficient HPCs}\label{M}
\subfile{contents/memory.tex}

\subsection{Fault Localization}
\subfile{contents/time_series.tex}

\section{Conclusion and Looking Ahead}
\subfile{contents/conclusion.tex}


\bibliography{ref}
\bibliographystyle{IEEEtran}

\end{document}
